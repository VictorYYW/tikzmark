%
% Removed from tikzmark.dtx for the present
%

% \section{Highlighting}
%
% \iffalse
%<*highlight>
% \fi
%
% From \url{http://tex.stackexchange.com/q/46434/86}
%
% The user-level start and stop commands.
%    \begin{macrocode}
\newcounter{highlight}
\newcommand{\hlstart}[1][]{%
  \hl@draw{#1}{highlighter}{0,0}{pic cs:hl-\the\value{highlight}}}

\newcommand{\hlend}{%
  \tikzmark{hl-\the\value{highlight}}\stepcounter{highlight}}

\newcommand{\fdstart}[1][]{%
    \def\fd@args{#1}%
    \tikzmark{hl-\the\value{highlight}}}

\newcommand{\fdend}{%
    \expandafter\hl@draw\expandafter{\fd@args}{fader}%
{pic cs:hl-\the\value{highlight}}{0,0}%
    \stepcounter{highlight}\def\fd@args{}}

\newcommand{\vlstart}[1][]{%
  \vl@draw{#1}{highlighter}{0,0}{pic cs:hl-\the\value{highlight}}}

\newcommand{\vlend}{%
  \tikzmark{hl-\the\value{highlight}}\stepcounter{highlight}}

\newcommand{\hlboxstart}[1][]{%
  \box@draw{#1}{highlighter}{0,0}{pic cs:hl-\the\value{highlight}}}

\newcommand{\hlboxend}{%
  \tikzmark{hl-\the\value{highlight}}\stepcounter{highlight}}
%    \end{macrocode}
%
% The command that draws the horizontal highligher or fader.
% This fills a shape determined by two coordinates assumed to be (in effect) on the baseline of the start and end of the region to be highlighted.
%    \begin{macrocode}
\def\hl@draw#1#2#3#4{%
    \begin{tikzpicture}[
      remember picture,
      overlay,
      baseline=0pt,
      /tikz/highlighter/.cd,
      #1,
      /tikz/.cd,
      highlight=#2,
      every path/.append style={
        highlight=#2
      }
    ]%
    \tikz@scan@one@point\pgfutil@firstofone(#3)\relax
    \pgf@ya=\pgf@y
    \tikz@scan@one@point\pgfutil@firstofone(#4)\relax
    \pgf@yb=\pgf@y
    \ifdim\pgf@ya=\pgf@yb
    \path (#3) ++(-1*\pgfkeysvalueof{/tikz/highlighter/initial offset},\pgfkeysvalueof{/tikz/highlighter/initial height}) coordinate (start);
    \path (#4) ++(\pgfkeysvalueof{/tikz/highlighter/final offset},-1*\pgfkeysvalueof{/tikz/highlighter/final depth}) coordinate (end);
    \fill (start) rectangle (end);
    \else
    \page@node
    \path (page.east) ++(\pgfkeysvalueof{/tikz/highlighter/right margin},0pt) coordinate (east);
    \path (page.west) ++(-1*\pgfkeysvalueof{/tikz/highlighter/left margin},0pt) coordinate (west);
  \pgfmathsetlength\pgf@x{\pgfkeysvalueof{/tikz/highlighter/initial height}}%
    \advance\pgf@yb by \pgf@x\relax
  \pgfmathsetlength\pgf@x{-1*\pgfkeysvalueof{/tikz/highlighter/final depth}}%
    \advance\pgf@ya by \pgf@x\relax
    \ifdim\pgf@yb>\pgf@ya
    \path (#3) ++(-1*\pgfkeysvalueof{/tikz/highlighter/initial offset},\pgfkeysvalueof{/tikz/highlighter/initial height}) coordinate (start);
    \path (#3) ++(0pt,-1*\pgfkeysvalueof{/tikz/highlighter/final depth}) coordinate (end);
    \fill (start) rectangle (end -| east);
    \path (#4) ++(0pt,\pgfkeysvalueof{/tikz/highlighter/initial height}) coordinate (start);
    \path (#4) ++(\pgfkeysvalueof{/tikz/highlighter/final offset},-1*\pgfkeysvalueof{/tikz/highlighter/final depth}) coordinate (end);
    \fill (start -| west) rectangle (end);
    \else
    \path (#3) ++(-1*\pgfkeysvalueof{/tikz/highlighter/initial offset},\pgfkeysvalueof{/tikz/highlighter/initial height}) coordinate (tl);
    \path (#3) ++(-1*\pgfkeysvalueof{/tikz/highlighter/initial offset},-1*\pgfkeysvalueof{/tikz/highlighter/initial depth}) coordinate (start);
    \path (#4) ++(\pgfkeysvalueof{/tikz/highlighter/final offset},-1*\pgfkeysvalueof{/tikz/highlighter/final depth}) coordinate (end);
    \path (#4) ++(\pgfkeysvalueof{/tikz/highlighter/final offset},\pgfkeysvalueof{/tikz/highlighter/final height}) coordinate (mr);
    \fill (start) -- (tl) -- (tl -| east) -- (mr -| east) -- (mr) -- (end) -- (end -| west) -- (start -| west) -- cycle;
    \fi
    \fi
    \end{tikzpicture}}
%    \end{macrocode}
%
% This one draws a box.
%    \begin{macrocode}
\def\box@draw#1#2#3#4{%
    \begin{tikzpicture}[
      remember picture,
      overlay,
      baseline=0pt,
      /tikz/highlighter/.cd,
      #1,
      /tikz/.cd,
      highlight=#2,
      every path/.append style={
        highlight=#2
      }
    ]%
   \path (#3) ++(-1*\pgfkeysvalueof{/tikz/highlighter/initial offset},\pgfkeysvalueof{/tikz/highlighter/initial height}) coordinate (start);
    \path (#4) ++(\pgfkeysvalueof{/tikz/highlighter/final offset},-1*\pgfkeysvalueof{/tikz/highlighter/final depth}) coordinate (end);
    \fill (start) rectangle (end);
    \end{tikzpicture}}
%    \end{macrocode}
%
% In this one the region is defined vertically.
%    \begin{macrocode}
\def\vl@draw#1#2#3#4{%
    \begin{tikzpicture}[
      remember picture,
      overlay,
      baseline=0pt,
      /tikz/highlighter/.cd,
      #1,
      /tikz/.cd,
      highlight=#2,
      every path/.append style={
        highlight=#2
      }
    ]%
    \tikz@scan@one@point\pgfutil@firstofone(#3)\relax
    \pgf@xa=\pgf@x
    \tikz@scan@one@point\pgfutil@firstofone(#4)\relax
    \pgf@xb=\pgf@x
    \ifdim\pgf@xa=\pgf@xb
    \path (#3) ++(\pgfkeysvalueof{/tikz/highlighter/initial height},\pgfkeysvalueof{/tikz/highlighter/initial offset}) coordinate (start);
    \path (#4) ++(-1*\pgfkeysvalueof{/tikz/highlighter/final depth},-1*\pgfkeysvalueof{/tikz/highlighter/final offset}) coordinate (end);
    \fill (start) rectangle (end);
    \else
    \page@node
    \path (page.north) ++(\pgfkeysvalueof{/tikz/highlighter/top margin},0) coordinate (north);
    \path (page.south) ++(-1*\pgfkeysvalueof{/tikz/highlighter/bottom margin},0) coordinate (south);
  \pgfmathsetlength\pgf@y{\pgfkeysvalueof{/tikz/highlighter/initial height}}
    \advance\pgf@xa by \pgf@y\relax
  \pgfmathsetlength\pgf@y{-1*\pgfkeysvalueof{/tikz/highlighter/final depth}}
    \advance\pgf@xb by \pgf@y\relax
    \ifdim\pgf@xb<\pgf@xa
    \path (#3) ++(\pgfkeysvalueof{/tikz/highlighter/initial height},\pgfkeysvalueof{/tikz/highlighter/initial offset}) coordinate (start);
    \path (#3) ++(-1*\pgfkeysvalueof{/tikz/highlighter/initial depth},0pt) coordinate (end);
    \fill (start) rectangle (end |- south);
    \path (#4) ++(\pgfkeysvalueof{/tikz/highlighter/final height},0) coordinate (start);
    \path (#4) ++(-1*\pgfkeysvalueof{/tikz/highlighter/final depth},-1*\pgfkeysvalueof{/tikz/highlighter/final depth}) coordinate (end);
    \fill (start) rectangle (end |- north);
    \else
    \path (#3) ++(-1*\pgfkeysvalueof{/tikz/highlighter/initial depth},\pgfkeysvalueof{/tikz/highlighter/initial offset}) coordinate (start);
    \path (#3) ++(\pgfkeysvalueof{/tikz/highlighter/initial height},\pgfkeysvalueof{/tikz/highlighter/initial offset}) coordinate (tl);
    \path (#4) ++(-1*\pgfkeysvalueof{/tikz/highlighter/final depth},-1*\pgfkeysvalueof{/tikz/highlighter/final offset}) coordinate (mr);
    \path (#4) ++(\pgfkeysvalueof{/tikz/highlighter/final height},-1*\pgfkeysvalueof{/tikz/highlighter/final offset}) coordinate (end);
    \fill (start) -- (tl) -- (tl |- north) -- (end |- north) -- (end) -- (mr) -- (mr |- south) -- (start |- south) -- cycle;
    \fi
    \fi
    \end{tikzpicture}}
%    \end{macrocode}
%
% These set various options.
%    \begin{macrocode}
\tikzset{%
  highlight/.default=highlighter,
  highlight/.style={
    every #1/.try,
    color=\pgfkeysvalueof{/tikz/#1/colour},
    line width=\pgfkeysvalueof{/tikz/#1/width},
    line cap=\pgfkeysvalueof{/tikz/#1/cap},
    opacity=\pgfkeysvalueof{/tikz/#1/opacity},
  },
  /tikz/highlighter/.is family,
  /tikz/highlighter/.unknown/.code={%
    \let\tk@searchname=\pgfkeyscurrentname%
    \pgfkeysalso{%
      /tikz/\tk@searchname=#1
    }
  },
  /tikz/highlighter/.cd,
  colour/.initial=yellow,
  width/.initial=12pt,
  cap/.initial=butt,
  opacity/.initial=1,
  initial height/.initial=\baselineskip,
  initial depth/.initial=.5ex,
  initial offset/.initial=.5\baselineskip,
  final height/.initial=\baselineskip,
  final depth/.initial=.5ex,
  final offset/.initial=.5\baselineskip,
  height/.style={
    initial height=#1,
    final height=#1
  },
  depth/.style={
    initial depth=#1,
    final depth=#1
  },
  offset/.style={
    initial offset=#1,
    final offset=#1
  },
  margin/.style={
    left margin=#1,
    right margin=#1,
    top margin=#1,
    bottom margin=#1,
  },
  left margin/.initial=.5\baselineskip,
  right margin/.initial=.5\baselineskip,
  top margin/.initial=.5\baselineskip,
  bottom margin/.initial=-.5\baselineskip,
  /tikz/fader/.is family,
  /tikz/fader/.cd,
  colour/.initial=gray,
  width/.initial=12pt,
  cap/.initial=butt,
  opacity/.initial=.5,
}
%    \end{macrocode}
%
% Some beamer specifics.
%    \begin{macrocode}
\@ifclassloaded{beamer}{

\setbeamercolor{highlighted text}{bg=yellow}
\setbeamercolor{faded text}{fg=gray}

\newcommand<>{\highlight}[2][]{%
  \only#3{\hlstart[#1]}#2\only#3{\hlend}}

\newcommand<>{\fade}[2][]{%
  \only#3{\fdstart[#1]}#2\only#3{\fdend}}

\newcommand<>{\vhighlight}[2][]{%
  \only#3{\vlstart[#1]}#2\only#3{\vlend}}

\def\page@node{
  \path (current page.south east)
      ++(-\beamer@rightmargin,\footheight)
  node[
    minimum width=\textwidth,
    minimum height=\textheight,
    anchor=south east
  ] (page) {};
}

}{
%    \end{macrocode}
% The non-beamer variants
%    \begin{macrocode}
  \def\page@node{
    \path (current page.north west)
    ++(\hoffset + 1in + \oddsidemargin + \leftskip,-\voffset - 1in - \topmargin - \headheight - \headsep)
    node[
      minimum width=\textwidth - \leftskip - \rightskip,
      minimum height=\textheight,
      anchor=north west,
      line width=0mm,
    ] (page) {};
  }

\newcommand{\highlight}[2][]{%
\hlstart[#1]#2\hlend}

\newcommand{\fade}[2][]{%
\fdstart[#1]#2\fdend}

\newcommand{\vhighlight}[2][]{%
\vlstart[#1]#2\vlend}

}
%    \end{macrocode}
%
%
%
%
% \iffalse
%</highlight>
% \fi
