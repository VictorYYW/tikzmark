\immediate\write18{tex tikzmark.dtx}
\documentclass{article}
\usepackage{tikz}
\usetikzlibrary{tikzmark}
\usetikzmarkextra{highlight}

\tikzset{
  next page=below,
  highlighter/right margin extent=2mm,
  highlighter/left margin extent=-2mm,
  fader/colour=gray,
  fader/opacity=.8,
%  every highlighter/.style={rounded corners},
}

\usepackage{lipsum}

\begin{document}

\lipsum[6]

\vlstart\lipsum*[5]\vlend

\lipsum[7]

\newpage

\hlboxstart\lipsum*[4]\hlboxend

\hlstart\lipsum*[1]\hlend

\hlstart\lipsum*[2]\hlend

\fdstart\lipsum*[3]\fdend

\hlstart A not so long sentence\hlend

A rather longer \hlstart sentence that goes over one line and onto the next but not all that much further\hlend\ than that.

A rather longer sentence that goes over one line and onto the next but not all that much further than that.

\tikz[overlay,remember picture,ultra thick,red] \draw (pic cs:a) -- (pic cs:b);%
A\tikzmark{a} tikzmark

\lipsum[1-4]
\lipsum[4]
\lipsum*[4]
\tikzmark{b}%

\lipsum[1]
\tikz[overlay,remember picture,ultra thick,red] \draw (pic cs:a) -- (pic cs:b);
\end{document}

\documentclass{article}
\usepackage{tikz}
\usetikzlibrary{tikzmark}

\tikzset{
  next page=below,
}

\usepackage{lipsum}

\begin{document}

\tikz[overlay,remember picture,ultra thick,red] \draw (pic cs:a) -- (pic cs:b);
\tikz[overlay,remember picture,ultra thick,red] \draw (0,0) -- (pic cs:d);

A\tikzmark{a} tikzmark \tikzmark{b}.

\lipsum[1-4]
\lipsum[4]
\tikzmark{c}%
\lipsum*[4]
\tikzmark{d}%

\lipsum[1]
\tikz[overlay,remember picture,ultra thick,red] \draw (0,0) -- (pic cs:d);

\end{document}

% Local Variables:
% tex-output-type: "pdf18"
% End:%   